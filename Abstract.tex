\chapter*{Abstract}
\mtcaddchapter[Abstract]
\label{chap:Abstract}
Low visibility has been a challenge for weather forecasters for a long time, due to its negative impact on air, sea and road traffic. Indeed, the human and financial losses attributable to reduced visibility become increasingly important, hence a good forecast of horizontal visibility is of great benefit to meteorological forecasters. So, to cope with this challenge in the area of numerical weather prediction, the potential of Datamining techniques to estimate horizontal visibility has been evaluated in several scientific studies. However, the performance of developed models differs from one study to another due to the variety of Datamining tools and algorithms used. Therefore, the objective of our study is to evaluate the sensitivity of the performance of the developed models to the Datamining platform and algorithm for a regression case which aims to estimate the visibility from the predictions of the operational numerical weather prediction AROME model. To achieve this goal, we used two types of algorithms : those based on the ensemble methods including \textit{gradient boosting}, \textit{eXtreme gradient boosting} and \textit{random forest}, and those based on \textit{deep learning}. The algorithms are evaluated under various open source platforms (\textit{Scikit-learn}, \textit{H2O}, \textit{WEKA}, \textit{Tensorflow} and \textit{Keras}). In addition, a database covering 3-year of hourly data, and resulting from preprocessing of the raw outputs of AROME and the observed data, was used in this work. The sampling of these data in 70\% for learning and 30\% for testing was carried out by guaranteeing the representativity of the months, the hours and the various classes of visibilities for all the synoptic stations. The results show that the performance of the models based on the ensemble methods is the best whatever the used platform except for \textit{Keras} where only the \textit{Deep Learning} was used. On the other hand, the \textit{Random Forest} algorithm is found as the best estimator of visibility after tuning of hyperparameters for the \textit{WEKA} and \textit{Scikit-learn} platforms. However, \textit{Gradient boosting machine} outperforms the other algorithms for the \textit{H2O} platform. Besides, the mean squared errors are 1933 m, 1942 m and 1945 m respectively for \textit{Gradient Boosting} under \textit{H2O}, \textit{Random Forest} for \textit{Scikit-learn} and \textit{WEKA}. Similarly for the mean absolute error took the following values 1199 m, 1221 m and 1232 m for the same algorithms and platforms.